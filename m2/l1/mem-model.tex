\subsection{Memory model}

\begin{frame}[t]{C++ and memory consistency}
\begin{itemize}
  \item \textgood{C++11} define its own \textmark{concurrency model} 
        as part of the language.

  \mode<presentation>{\vfill}
  \item \textgood{Goal}: 
        Avoid the need for coding in
        \textmark{lower level languages} (C, assembly, \ldots) 
        to get \textgood{better performance}.
    \begin{itemize}
      \item \textemph{Atomic} types.
      \item Low level \textemph{synchronization} mechanisms.
    \end{itemize}

  \mode<presentation>{\vfill}
  \item Allows \textmark{creation} of
        \textgood{lock free} data structures.
\end{itemize}
\end{frame}

\begin{frame}[t]{Objects and memory locations}
\begin{itemize}
  \item \textgood{Object}: 
        A storage region.
    \begin{itemize}
      \item A sequence of one or more bytes.
    \end{itemize}

  \mode<presentation>{\vfill}
  \item \textgood{Memory location}: An object of a \textmark{scalar type} 
        or a sequence of \textmark{adjacent bit fields}.

  \mode<presentation>{\vfill}
  \item An \textgood{object} is stored in one or several
        \textmark{memory locations}.
\end{itemize}
\end{frame}

\begin{frame}[t,fragile]{Example}
\begin{itemize}
  \item Structure:
\begin{lstlisting}
struct S {
  int a;
  char b;
  int c: 10;
  int d: 16;
  double e;
};
S s;
\end{lstlisting}
  \mode<presentation>{\vfill}
  \item Memory locations:
    \begin{enumerate}
      \item \cppid{s.a}.
      \item \cppid{s.b}.
      \item \cppid{s.c}, \cppid{s.d}.
      \item \cppid{s.e}.
    \end{enumerate}
\end{itemize}
\end{frame}

\begin{frame}[t]{Rules}
\begin{itemize}
  \item \textgood{Two threads} may access to different \textmark{memory locations}
        simultaneously.

  \mode<presentation>{\vfill\pause}
  \item \textgood{Two threads} may access the \textmark{same memory location}
        simultaneously if both are read access. 

  \mode<presentation>{\vfill\pause}
  \item If \textgood{two threads} try to access simultaneously to the same
        \textmark{memory location} and some the accesses is a \textemph{write}
        there is a \textbad{potential race condition}.
    \begin{itemize}
      \item Depends whether \textemph{an ordering is ensured} between both accesses.
    \end{itemize}
\end{itemize}
\end{frame}

\begin{frame}[t]{Ordering and races}
\begin{itemize}
  \item \textgood{Classic solution}: 
        Using synchronization mechanisms.
    \begin{itemize}
      \item Allows to guarantee mutual exclusion.
      \item OS based $\rightarrow$ might be costly.
    \end{itemize}

  \mode<presentation>{\vfill\pause}
  \item \textgood{Alternative}: Using \textmark{atomic operations} to
        ensure \textemph{ordering} 

  \mode<presentation>{\vfill\pause}
  \item If \textmark{an ordering is not ensured} between two accesses to a memory location,
        any of the accesses is \textmark{non-atomic}, 
        and at least one of the accesses is a \textmark{write},
        those constitute a \textbad{data race} and
        \textbad{program behavior is undefined}.
\end{itemize}
\end{frame}

\begin{frame}[t]{Modification order}
\begin{itemize}
  \item Sequence of writes to an object.
    \begin{itemize}
      \item If two threads see \textmark{different modification orders} on an object
            there is a \textbad{data race}.
      \item \textgood{Modifications} do not need to be \textmark{visible}
            at the same time in all threads.
    \end{itemize}

  \mode<presentation>{\vfill\pause}
  \item A \textmark{read that is subsequent to a write}
        in the same thread observes the \textgood{written value}
        or a \textgood{later value} in its
        \textemph{modification order}.
\end{itemize}
\end{frame}

