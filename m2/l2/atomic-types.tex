\subsection{Atomic types}

\begin{frame}[t]{Atomic operations}
\begin{itemize}
  \item They are \textemph{indivisible operations}.

  \mode<presentation>{\vfill\pause}
  \item If a \textgood{thread} performs and \textmark{atomic read} on a variable and
        another \textgood{thread} performs and \textmark{atomic write} on the same variable
        and there are \textemph{no other threads} accessing:

    \mode<presentation>{\vfill\pause}
    \begin{itemize}
      \item The \textgood{read} returns \textmark{value previous} to the write
            or the \textmark{written value}.

      \mode<presentation>{\vfill\pause}
      \item If \textgood{any of the operations} (read or write) is 
            \textmark{non atomic}, \textbad{behavior is undefined}.
        \begin{itemize}
          \item Read value might be different from previous and subsequent ones.
        \end{itemize}
    \end{itemize}
\end{itemize}
\end{frame}

\begin{frame}[t]{Atomic types}
\begin{itemize}
  \item Generic type \cppid{atomic<T>} allows defining atomic variables
        for type \cppid{T}, where \cppid{T} is:
    \begin{itemize}
      \item An \textmark{integral}.
      \item A \textmark{pointer type}.
      \item \cppid{bool}.
      \item \textbad{Limited interface} for \textmark{floating point types} 
            (\cppid{float}, \cppid{double}, \cppid{long double}).
      \item Also for \textmark{user defined types} with some \textbad{constraints}.
        \begin{itemize}
          \item Trivially copyable, copy/move constructible/assignable.
        \end{itemize}
    \end{itemize}

  \mode<presentation>{\vfill\pause}
  \item All \textemph{atomic types} have a member \cppid{is\_lock\_free()}.
    \begin{itemize}
      \item Determines whether the implementation is \textmark{lock free}.
    \end{itemize}

  \mode<presentation>{\vfill\pause}
  \item Additionally, type \cppid{atomic\_flag}:
    \begin{itemize}
      \item The only one guaranteed to be \textmark{lock free}.
    \end{itemize}
\end{itemize}
\end{frame}

\begin{frame}[t,fragile]{\texttt{atomic\_flag}}
\begin{itemize}
  \item \textmark{Simplest} \textgood{atomic type}.
    \begin{itemize}
      \item Two possible \textenum{states}: 
            \textemph{enabled} or \textemph{disabled}.

      \mode<presentation>{\vfill\pause}
      \item It is always \textmark{lock free}.

      \mode<presentation>{\vfill\pause}
      \item Always \textmark{explictly initialized} to \textemph{disabled}.
\begin{lstlisting}
std::atomic_flag f1 = ATOMIC_FLAG_INIT;
\end{lstlisting}

      \mode<presentation>{\vfill\pause}
      \item \textenum{Operations}:
        \begin{itemize}
          \item Disable: 
\begin{lstlisting}
f1.clear();
\end{lstlisting}

          \item Enable and check previous value: 
\begin{lstlisting}
f1.test_and_set();
\end{lstlisting}
        \end{itemize}

      \mode<presentation>{\vfill\pause}
      \item A \textmark{memory ordering} may be supplied.
    \end{itemize}
\end{itemize}
\end{frame}

\begin{frame}[t,fragile]{Example: A \emph{spin lock}}
\begin{itemize}
  \item A \textgood{spin lock} is a \textmark{lock} 
        not making use of OS services.
    \begin{itemize}
      \item Useful when you know blocking is very short and you want to avoid
            context switching issues.
    \end{itemize}
\mode<presentation>{\vfill\pause}
\begin{lstlisting}
class spinlock_mutex {
private:
  std::atomic_flag f;
public:
  spinlock_mutex() : f{ATOMIC_FLAG_INIT} {}

  void lock() {
    while (f.test_and_set()) {}
  }

  void unlock() {
    f.clear();
  }
};
\end{lstlisting}
\end{itemize}
\end{frame}

\begin{frame}[t,fragile]{\texttt{atomic<bool>}}
\mode<presentation>{\vspace{-1em}}
\begin{itemize}
  \item More operations than \cppid{atomic\_flag}.
    \item May be initialized and assigned with \cppid{bool}s.
    \item Cannot be copied from another \cppid{atomic<bool>}.
    \item Modification: 
\begin{lstlisting}
a.store(b); // b is a bool
\end{lstlisting}
    \item Query:
      \begin{itemize}
        \item Load a value:
\begin{lstlisting}
a.load()
\end{lstlisting}
        \item Swapping:
\begin{lstlisting}
old = a.exchange(b); // b and old are bools
\end{lstlisting}
        \item Automatic conversion to \cppid{bool} (sequential consistency)
      \end{itemize}
\end{itemize}
\begin{lstlisting}
std::atomic<bool> a;
bool x = a.load();
a.store(true);
x = a.exchange(false);
\end{lstlisting}
\end{frame}

\begin{frame}[t,fragile]{Compare and swap}
\begin{itemize}
  \item Compares the value of the \textgood{atomic} with an 
        \textmark{expected value}.
    \begin{itemize}
      \item If equal, exchange the atomic and \textmark{desired}:
         \begin{itemize}
           \item Stores the \textmark{desired} value in the atomic.
           \item The atomic value is stored into the \textmark{expected}.
         \end{itemize}
      \item If not equal, load atomic into \textmark{desired}:
        \begin{itemize}
          \item The atomic is left unmodified.
           \item The atomic value is stored into the \textmark{expected}.
        \end{itemize}
      \item It always returns indication of \textemph{success} or \textemph{failure}.
    \end{itemize}

  \mode<presentation>{\vfill\pause}
  \item \textenum{Two versions}:

\mode<presentation>{\vfill}
\begin{lstlisting}
a.compare_exchange_weak(e,d)
\end{lstlisting}
    \begin{itemize}
      \item Allows \textmark{spurious failures} (context switch) in some architectures.
    \end{itemize}

\mode<presentation>{\vfill}
\begin{lstlisting}
a.compare_exchange_strong(e,d)
\end{lstlisting}
    \begin{itemize}
      \item Does not allow for \textmark{spurious failures}.
    \end{itemize}
\end{itemize}
\end{frame}

\begin{frame}[t,fragile]{\texttt{atomic<void*>}}
\begin{itemize}
  \item Atomic access to memory address.
  \item Cannot be copied.
  \item Can copy a pointer \cppid{void*}.
  \item Interface similar to \cppid{atomic<bool>}:
    \begin{itemize}
      \item \cppid{is\_lock\_free()}, \cppid{load()}, \cppid{store()}, \cppid{exchange()},
            \cppid{compare\_exchange\_weak/strong()}.
    \end{itemize}
  \item Additional operations.
    \begin{itemize}
      \item \cppid{fetch\_add()}, \cppid{fetch\_sub()}.
        \begin{itemize}
          \item Ordering may be supplied.
          \item Return value before the update.
        \end{itemize}
    \end{itemize}
      \item \cppid{+=}, \cppid{-=}.
        \begin{itemize}
          \item Return value after the update.
        \end{itemize}
      \item Always use byte arithmetic.
      \item Other arithmetics: \cppid{atomic<T*>}.
\end{itemize}
\end{frame}

\begin{frame}[t,fragile]{\texttt{atomic<integral>}}
\begin{itemize}
  \item Can be applied to all integral types.
  \item General operations:
    \begin{itemize}
      \item \cppid{is\_lock\_free()}, \cppid{load()}, \cppid{store()}, \cppid{exchange()}, 
            \cppid{compare\_exchange\_weak/strong()}.
    \end{itemize}
  \item Arithmetic operations
    \begin{itemize}
      \item \cppid{fetch\_add()}, \cppid{fetch\_sub()}, \cppid{fetch\_and()}, \cppid{fetch\_or()}, \cppid{fetch\_xor()}.
      \item \cppid{+=}, \cppid{-=}, \cppid{\&=}, \cppid{|=}, \cppid{\^{}=}.
      \item \cppid{++x}, \cppid{x++}, \cppid{--x}, \cppid{x--}
      \item No other arithmetic operations: (\cppid{*}, \cppid{/}, \cppid{\%}).
    \end{itemize}
\end{itemize}
\end{frame}


