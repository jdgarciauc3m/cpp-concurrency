\subsection{Ordering relationships}

\begin{frame}[t]{\emph{synchronizes-with} relationship}
\begin{itemize}
  \item Relationship between \textmark{operations} on \textgood{atomic types}.

  \mode<presentation>{\vfill\pause}
  \item A \textmark{write} on an \textgood{atomic value} 
        \textgood{synchronizes with} a \textmark{read} on that
        \textgood{atomic value} that \textmark{reads the value}:
    \begin{itemize}

      \mode<presentation>{\vfill\pause}
      \item Stored by that \textmark{write}.

      \mode<presentation>{\vfill\pause}
      \item Stored by a \textmark{subsequent write} from the
            \textemph{same thread} that performed the write.

      \mode<presentation>{\vfill\pause}
      \item Stored by a \textmark{sequence of operations} 
            \textgood{read-modify-write} on the value from
            \textemph{any thread} where the \textmark{first operation} 
            read the value stored by the write.
    \end{itemize}
\end{itemize}
\end{frame}


\begin{frame}[t]{\emph{happens-before} relationship}
\begin{itemize}
  \item Specifies \textgood{which operation} sees the \textmark{effects from other operation}.

  \mode<presentation>{\vfill\pause}
  \item Within a thread,
        an operation \textgood{happens-before} than other operation
        if it appears in a previous statement.

  \mode<presentation>{\vfill\pause}
  \item There is no ordering between two operations in the same statement.

  \mode<presentation>{\vfill\pause}
  \item Among two threads,
        an operation from one thread \textgood{happens-before} than other operation in
        another thread if:
    \begin{itemize}
      \item There is a \textmark{synchronizes-with} relationship among them.
      \item There is a chain of \textmark{happens-before} and \textmark{synchronizes-with} 
            relationships between them.
    \end{itemize}
\end{itemize}
\end{frame}

\begin{frame}[t,fragile]{Example}
\begin{columns}
\begin{column}{.5\textwidth}
\begin{block}{Writer/reader}
\begin{lstlisting}
std::vector<int> v; 
std::atomic<bool> f{false}; 

void writer() { 
  v.push_back(1);
  f = true;
}

void reader() { 
  while(!f.load()) {
    std::this_thread::sleep_for(
      std::milliseconds{1}); 
  }
  std::cout << v[0] << "\n";
} 
\end{lstlisting}
\end{block}
\end{column}
\begin{column}{.5\textwidth}
\begin{tikzpicture}
\tikzset{
    sentencia/.style={rectangle,rounded corners,draw=black, top color=white, bottom color=blue!50,very thick, inner sep=0.5em, minimum size=0, text centered, font=\tiny},
    flecha/.style={->, >=latex', shorten >=1pt, thick, draw=blue},
    flechasinc/.style={->, >=latex', shorten >=1pt, thick, draw=red},
    etiqueta/.style={text centered, font=\tiny} 
}  
\pause
\node[sentencia] (pushback) {v.push\_back(1);};
\node[sentencia, below=1cm of pushback] (ftrue) {f=true;};
\draw[flecha] (pushback) -- (ftrue);
\node[sentencia, right=1cm of pushback] (fload1) {f.load()};
\node[etiqueta, right=0.25cm of fload1] (etfload1) {false};
\node[sentencia, below=0.5cm of fload1] (fload2) {f.load()};
\node[etiqueta, right=0.25cm of fload2] (etfload2) {true};
\draw[flecha] (fload1) -- (fload2);
\node[sentencia, below=1cm of fload2] (print) {std::cout <{}< v[0]; };
\draw[flecha] (fload2) -- (print);
\pause
\draw[flechasinc] (ftrue) -- (fload2);
\onslide<1->
\end{tikzpicture}
\end{column}
\end{columns}
\end{frame}

