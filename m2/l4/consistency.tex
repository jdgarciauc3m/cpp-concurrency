\subsection{Consistency models}

\begin{frame}[t]{Sequential consistency}
\begin{itemize}
  \item \cppid{memory\_order\_seq\_cst}.
  \mode<presentation>{\vfill}
  \item The program is consistent with a sequential view.
  \mode<presentation>{\vfill}
  \item If all operations on atomics are sequentially consistent,
    \begin{itemize}
      \item the multi-threaded program behavior is as if all operations
            were performed in some particular order in a single thread.
    \end{itemize}
  \mode<presentation>{\vfill}
  \item \textenum{Consequences}:
    \begin{itemize}
      \item There cannot be reorderings.
      \item It is the simplest model to reason about.
      \item It is the most expensive model in terms of performance.
    \end{itemize}
\end{itemize}
\end{frame}

\begin{frame}[t,fragile]{Example}
\begin{lstlisting}
std::atomic<bool> x, y;
std::atomic<int> z;

void f() {
  x.store(true, std::memory_order_seq_cst);
}

void g() {
  y.store(true, std::memory_order_seq_cst);
}

void h() {
  while (!x.load(std::memory_order_seq_cst)) {}
  if (y.load(std::memory_order_seq_cst))  ++ z;
}

void i() {
  while (!y.load(std::memory_order_seq_cst)) {}
  if (x.load(std::memory_order_seq_cst)) ++z;
}
\end{lstlisting}
\end{frame}

\begin{frame}[t,fragile]{Example}
\begin{lstlisting}
int main() {
  x = false;
  y = false;
  z = 0;
  std::thread t1{f};
  std::thread t2{g};
  std::thread t3{h};
  std::thread t4{i};
  t1.join();
  t2.join();
  t3.join();
  t4.join();
  assert(z.load() !=0);
  return 0;
}
\end{lstlisting}
\end{frame}

\begin{frame}[t,fragile]{Sequential consistency}
\begin{tikzpicture}
\tikzset{
    sentencia/.style={rectangle,rounded corners,draw=black, top color=white, bottom color=blue!50,very thick, inner sep=0.5em, minimum size=0, text centered, font=\tiny},
    flecha/.style={->, >=latex', shorten >=1pt, thick, draw=blue},
    flechasinc/.style={->, >=latex', shorten >=1pt, thick, draw=red},
    etiqueta/.style={text centered, font=\tiny} 
}  
%\pause
\node[sentencia] (xstore) {x.store(true)};
\node[sentencia,right=1cm of xstore] (xload1) {x.load()};
\node[etiqueta, right=0.1cm of xload1] (xload1et) {false};
\node[sentencia,below=1cm of xload1] (xload2) {x.load()};
\node[etiqueta, right=0.1cm of xload2] (xload2et) {true};
\draw[flecha] (xload1) -- (xload2);
\node[sentencia,right=1cm of xload1et] (yload1) {y.load()};
\node[etiqueta, right=0.1cm of yload1] (xload1et) {false};
\node[sentencia,right=1cm of xload1et] (ystore) {y.store(true)};
\node[sentencia,below=1cm of xload2] (ifyload) {y.load()};
\draw[flecha] (xload2) -- (ifyload);
\node[sentencia,below=1cm of yload1] (yload2) {y.load()};
\node[etiqueta, right=0.1cm of yload2] (yload2et) {true};
\draw[flecha] (yload1) -- (yload2);
\node[sentencia,below=1cm of yload2] (ifxload) {x.load()};
\draw[flecha] (yload2) -- (ifxload);
\pause
\draw[flechasinc] (xstore) -- (xload2);
\draw[flechasinc] (ystore) -- (yload2);
\pause
\node[etiqueta,right=0.1cm of ifyload, text=red] (ifyloadet) {false?};
\pause
\draw[flechasinc] (ifyload) to [bend right=80] (ystore);
\pause
\node[etiqueta,right=0.1cm of ifxload, text=red] (ifxloadet) {true};
\node[etiqueta,below=0.05cm of ifxloadet, text=red] (incz) {z++};
\onslide<1->
\end{tikzpicture}
\end{frame}

\begin{frame}[t,fragile]{Sequential consistency}
\begin{tikzpicture}
\tikzset{
    sentencia/.style={rectangle,rounded corners,draw=black, top color=white, bottom color=blue!50,very thick, inner sep=0.5em, minimum size=0, text centered, font=\tiny},
    flecha/.style={->, >=latex', shorten >=1pt, thick, draw=blue},
    flechasinc/.style={->, >=latex', shorten >=1pt, thick, draw=red},
    etiqueta/.style={text centered, font=\tiny} 
}  
%\pause
\node[sentencia] (xstore) {x.store(true)};
\node[sentencia,right=1cm of xstore] (xload1) {x.load()};
\node[etiqueta, right=0.1cm of xload1] (xload1et) {false};
\node[sentencia,below=1cm of xload1] (xload2) {x.load()};
\node[etiqueta, right=0.1cm of xload2] (xload2et) {true};
\draw[flecha] (xload1) -- (xload2);
\node[sentencia,right=1cm of xload1et] (yload1) {y.load()};
\node[etiqueta, right=0.1cm of yload1] (xload1et) {false};
\node[sentencia,right=1cm of xload1et] (ystore) {y.store(true)};
\node[sentencia,below=1cm of xload2] (ifyload) {y.load()};
\draw[flecha] (xload2) -- (ifyload);
\node[sentencia,below=1cm of yload1] (yload2) {y.load()};
\node[etiqueta, right=0.1cm of yload2] (yload2et) {true};
\draw[flecha] (yload1) -- (yload2);
\node[sentencia,below=1cm of yload2] (ifxload) {x.load()};
\draw[flecha] (yload2) -- (ifxload);
\pause
\draw[flechasinc] (xstore) -- (xload2);
\draw[flechasinc] (ystore) -- (yload2);
\pause
\node[etiqueta,right=0.1cm of ifxload, text=red] (ifxloadet) {false?};
\pause
\draw[flechasinc] (ifxload) to [bend left=80] (xstore);
\pause
\node[etiqueta,right=0.1cm of ifyload, text=red] (ifyloadet) {true};
\node[etiqueta,below=0.05cm of ifyloadet, text=red] (incz) {z++};
\onslide<1->
\end{tikzpicture}
\end{frame}

\begin{frame}[t,fragile]{Non sequentially consistent orderings}
\begin{itemize}
\item No global ordering of events.
  \begin{itemize}
    \item Each thread might have a different view.
      \begin{itemize}
        \item Threads might not agree in event ordering.
      \end{itemize}
    \item But, \ldots
      \begin{itemize}
        \item All threads must agree in the modification order for each variable.
      \end{itemize}
  \end{itemize}
\mode<presentation>{\vfill}
\item \textenum{Alternatives}:
  \begin{itemize}
    \item Relaxed ordering.
    \item Release/Acquire ordering
  \end{itemize}
\end{itemize}
\end{frame}

\begin{frame}[t]{Relaxed ordering}
\begin{itemize}
  \item \cppid{memory\_order\_relaxed}
  \mode<presentation>{\vfill}
  \item Relaxed operations on atomics do not participate in \textmark{synchronizes-with}
        relationships.
  \mode<presentation>{\vfill}
  \item Operations on the same variable in the same thread follow \textmark{happens-before}
        relationships.
    \begin{itemize}
      \item Accesses to an atomic variable in the same thread cannot be reordered.
      \item Once a thread has seen a value from a variable,
            it cannot see an older value for that variable.
    \end{itemize}
\end{itemize}
\end{frame}

\begin{frame}[t,fragile]{Example}
\begin{columns}
\begin{column}{0.7\textwidth}
\begin{lstlisting}
std::atomic<bool> x, y; std::atomic<int> z;

void f() {
  x.store(true, std::memory_order_relaxed);
  y.store(true, std::memory_order_relaxed);
}
void g() {
  while (!y.load(std::memory_order_relaxed)) {}
  if (x.load(std::memory_order_relaxed)) ++z;
}

int main() {
  x = false;
  y = false;
  z = 0;
  std::thread t1{f}; std::thread t2{g};
  t1.join(); t2.join();
  assert(z.load() !=0);
  return 0;
}
\end{lstlisting}
\end{column}
\begin{column}{0.4\textwidth}
\begin{tikzpicture}
\tikzset{
    sentencia/.style={rectangle,rounded corners,draw=black, top color=white, bottom color=blue!50,very thick, inner sep=0.5em, minimum size=0, text centered, font=\tiny},
    flecha/.style={->, >=latex', shorten >=1pt, thick, draw=blue},
    flechasinc/.style={->, >=latex', shorten >=1pt, thick, draw=red},
    etiqueta/.style={text centered, font=\tiny} 
}  
\pause
\node[sentencia] (xstore) {x.store(true);};
\node[sentencia,below=1cm of xstore] (ystore) {y.store(true);};
\draw[flecha] (xstore) -- (ystore);
\node[sentencia,right=1cm of xstore] (yload1) {y.load()};
\node[etiqueta, right=0.1cm of yload1] (xload1et) {false};
\node[sentencia,below=1cm of yload1] (yload2) {y.load()};
\node[etiqueta, right=0.1cm of yload2] (yload2et) {true};
\draw[flecha] (yload1) -- (yload2);
\node[sentencia,below=1cm of yload2] (ifxload) {x.load()};
\draw[flecha] (yload2) -- (ifxload);
\pause
\node[etiqueta,right=0.1cm of ifxload, text=red] (ifxloadet) {??};
\onslide<1->
\end{tikzpicture}
\end{column}
\end{columns}
\end{frame}

\begin{frame}[t]{Release/Acquire ordering}
\begin{itemize}
  \item \cppid{memory\_order\_acquire}, \cppid{memory\_order\_release}, \cppid{memory\_order\_acq\_rel}.
  \mode<presentation>{\vfill}
  \item Intermediate synchronization level.
  \mode<presentation>{\vfill}
  \item A release operation that writes a value,
        \textemph{synchronizes-with} an acquire operation that reads that value.
  \mode<presentation>{\vfill}
  \item \textenum{Impact}:
    \begin{itemize}
      \item Different threads can see different orders.
      \item Not all orders are possible.
    \end{itemize}
\end{itemize}
\end{frame}


\begin{frame}[t,fragile]{Example}
\begin{lstlisting}
std::atomic<bool> x, y;
std::atomic<int> z;

void f() {
  x.store(true, std::memory_order_release);
}

void g() {
  y.store(true, std::memory_order_release);
}

void h() {
  while (!x.load(std::memory_order_acquire)) {}
  if (y.load(std::memory_order_acquire))  ++ z;
}

void i() {
  while (!y.load(std::memory_order_acquire)) {}
  if (x.load(std::memory_order_acquire)) ++z;
}
\end{lstlisting}
\end{frame}


\begin{frame}[t,fragile]{Example}
\begin{lstlisting}
int main() {
  x = false;
  y = false;
  z = 0;
  std::thread t1{f};
  std::thread t2{g};
  std::thread t3{h};
  std::thread t4{i};
  t1.join();
  t2.join();
  t3.join();
  t4.join();
  assert(z.load() !=0);
  return 0;
}
\end{lstlisting}
\end{frame}

\begin{frame}[t,fragile]{Release/Acquire consistency}
\begin{tikzpicture}
\tikzset{
    sentencia/.style={rectangle,rounded corners,draw=black, top color=white, bottom color=blue!50,very thick, inner sep=0.5em, minimum size=0, text centered, font=\tiny},
    flecha/.style={->, >=latex', shorten >=1pt, thick, draw=blue},
    flechasinc/.style={->, >=latex', shorten >=1pt, thick, draw=red},
    etiqueta/.style={text centered, font=\tiny} 
}  
%\pause
\node[sentencia] (xstore) {x.store(true,release)};
\node[sentencia,right=1cm of xstore] (xload1) {x.load(acquire)};
\node[etiqueta, right=0.1cm of xload1] (xload1et) {false};
\node[sentencia,below=1cm of xload1] (xload2) {x.load(acquire)};
\node[etiqueta, right=0.1cm of xload2] (xload2et) {true};
\draw[flecha] (xload1) -- (xload2);
\node[sentencia,right=1cm of xload1] (yload1) {y.load(acquire)};
\node[etiqueta, right=0.1cm of yload1] (xload1et) {false};
\node[sentencia,right=1cm of yload1] (ystore) {y.store(true,release)};
\node[sentencia,below=1cm of xload2] (ifyload) {y.load(acquire)};
\draw[flecha] (xload2) -- (ifyload);
\node[sentencia,below=1cm of yload1] (yload2) {y.load(acquire)};
\node[etiqueta, right=0.1cm of yload2] (yload2et) {true};
\draw[flecha] (yload1) -- (yload2);
\node[sentencia,below=1cm of yload2] (ifxload) {x.load(acquire)};
\draw[flecha] (yload2) -- (ifxload);
\pause
\draw[flechasinc] (xstore) -- (xload2);
\draw[flechasinc] (ystore) -- (yload2);
\pause
\node[etiqueta,right=0.1cm of ifyload, text=red] (ifyloadet) {?};
\node[etiqueta,right=0.1cm of ifxload, text=red] (ifxloadet) {?};
\onslide<1->
\end{tikzpicture}
\pause
\begin{itemize}
  \item Multiple orderings are possible.
    \begin{itemize}
      \item There are no \emph{acquire} $\rightarrow$ \emph{release} relationships.
    \end{itemize}
\end{itemize}
\end{frame}

\begin{frame}[t,fragile]{Combining orderings}
\begin{small}
\begin{itemize}
  \item Effect equivalent to sequential consistency but with less cost.
\end{itemize}
\end{small}
\begin{columns}[T]
\begin{column}{.5\textwidth}
\begin{lstlisting}[basicstyle=\tiny]
std::atomic<bool> x, y; 
std::atomic<int> z;

void f() {
  x.store(true, std::memory_order_relaxed);
  y.store(true, std::memory_order_release);
}

void g() {
  while (!y.load(std::memory_order_acquire)) {}
  if (x.load(std::memory_order_relaxed)) ++z;
}

int main() {
  x = false; y = false; z = 0;
  std::thread t1{f}; std::thread t2{g};
  t1.join(); t2.join();
  assert(z.load() !=0);
  return 0;
}
\end{lstlisting}
\end{column}
\begin{column}{.5\textwidth}
\begin{tikzpicture}
\tikzset{
    sentencia/.style={rectangle,rounded corners,draw=black, top color=white, bottom color=blue!50,very thick, inner sep=0.5em, minimum size=0, text centered, font=\tiny},
    flecha/.style={->, >=latex', shorten >=1pt, thick, draw=blue},
    flechasinc/.style={->, >=latex', shorten >=1pt, thick, draw=red},
    etiqueta/.style={text centered, font=\tiny} 
}  
\pause
\node[sentencia] (xstore) {x.store(true,relaxed);};
\node[sentencia,below=1cm of xstore] (ystore) {y.store(true,release);};
\draw[flecha] (xstore) -- (ystore);
\node[sentencia,right=0.5cm of xstore] (yload1) {y.load(acquire)};
\node[etiqueta, right=0.1cm of yload1] (xload1et) {false};
\node[sentencia,below=1cm of yload1] (yload2) {y.load(acquire)};
\node[etiqueta, right=0.1cm of yload2] (yload2et) {true};
\draw[flecha] (yload1) -- (yload2);
\node[sentencia,below=1cm of yload2] (ifxload) {x.load(relaxed)};
\draw[flecha] (yload2) -- (ifxload);
\pause
\draw[flechasinc] (ystore) -- (yload2);
\pause
\node[etiqueta,right=0.1cm of ifxload, text=red] (ifxloadet) {true};
\onslide<1->
\end{tikzpicture}
\end{column}
\end{columns}
\end{frame}

